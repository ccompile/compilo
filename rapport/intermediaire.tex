\documentclass[a4paper]{article}

\usepackage[french]{babel}
\usepackage[T1]{fontenc}
\usepackage[utf8]{inputenc}
\usepackage{fullpage}
\usepackage{pgf}
\usepackage{tikz}
\usetikzlibrary{arrows,automata}
\usetikzlibrary{positioning}

\begin{document}

\title{Compilateur pour MiniC : analyse syntaxique et typage}
\author{Thomas Bourgeat et Antonin Delpeuch}
\date{\today}

\maketitle

\section{Architecture}

Nous avons fait le choix d'utiliser deux types d'arbres aux différents
stades de la compilation, le premier est crée au parsing, le deuxième au
typage :
\begin{itemize}
    \item Un arbre de syntaxe abstraite étiqueté par la position des
expressions dans le source initial.
    \item Un arbre sémantique étiqueté par les types de chaque
expression.
(pour les nœuds où cette information est utile dans la phase de
production de code)
\end{itemize}

\section{Analyse lexicale}

\section{Analyse syntaxique}

\section{Typage}

\appendix
\section{Conventions de nommage}

Les types commencent par une lettre indiquant dans quel contexte on les utilise :
\begin{itemize}
    \item \texttt{a} comme dans \texttt{aexpr} : type utilisé dans l'arbre de syntaxe abstraite pour représenter un nœud.
    \item \texttt{l} comme dans \texttt{lident} : équivalent étiqueté (par une position dans le source) du type précédent.
    On définit généralement \texttt{l\emph{type} = label * a\emph{type}}.
\item \texttt{w} comme dans \texttt{wexpr} : arbre dont la descendance est étiquetée par des types C
\item \texttt{t} comme dans \texttt{texpr} : équivalent étiqueté par un type C du type précédent, généralement défini par \texttt{t\emph{type} = expr\_type * w\emph{type}}.
\end{itemize}

Les constructeurs de chaque type sont préfixés par des majuscules indiquant quel type ils construisent :
\begin{itemize}
    \item \texttt{AU} : opérateur unaire dans un arbre de syntaxe étiqueté par des positions
    \item \texttt{AB} : idem pour un opérateur binaire
    \item \texttt{AE} : une expression dans l'AST (équivalent du \emph{<expr>} du sujet)
    \item \texttt{AI} : une instruction dans l'AST
    \item \texttt{ET} : un type qui peut être affecté à une expression (Expresion Type)
    \item \texttt{TE} : une expression typée (Typed Expression), pour construire l'AST typé
\end{itemize}

\end{document}


