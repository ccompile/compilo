\documentclass[a4paper]{article}

\usepackage[french]{babel}
\usepackage[T1]{fontenc}
\usepackage[utf8]{inputenc}
\usepackage{fullpage}
\usepackage{pgf}
\usepackage{tikz}
\usetikzlibrary{arrows,automata}
\usetikzlibrary{positioning}

\begin{document}

\title{Compilateur pour MiniC : analyse syntaxique et typage}
\author{Thomas Bourgeat et Antonin Delpeuch}
\date{\today}

\maketitle

\section{Architecture}

Nous avons fait le choix d'utiliser deux types d'arbres aux différents
stades de la compilation, le premier est crée au parsing, le deuxième au
typage :
\begin{itemize}
    \item Un arbre de syntaxe abstraite étiqueté par la position des
expressions dans le code source initial.
    \item Un arbre sémantique étiqueté par les types de chaque
expression.
(pour les nœuds où cette information est utile dans la phase de
production de code)
\end{itemize}

\section{Analyse lexicale}

Nous avons utilisé \texttt{ocamllex} comme conseillé.
Cette partie n'a pas été des plus créatives, nous avons essentiellement suivi les directives du cours (enfin, on espère !).
D'autre part,nous avons décidé d'implémenter dès le Lexing la transformation des
entiers exprimés en Octal/Hexa dans le code C en décimal. Enfin, il a
fallu utiliser le module \texttt{Int32} de OCaml pour avoir des entiers de 32
bits. En effet le type \texttt{int} d'ocaml ne fait que 31 bits signés,
ce qui aurait posé des problèmes de dépassement d'entiers.

\section{Analyse syntaxique}

\subsection{Construction de l'arbre}

Nous avons recontré deux problèmes majeurs en utilisant \texttt{menhir} :
\begin{itemize}
%RETEST UNE FOIS TERMINE: Fonctionne en rajoutant nonassoc INCR, donc
%le problème n'était pas exactement ça
    %\item Nous pensions que \texttt{++} et \texttt{--} étaient (à tord) des
    %opérateurs non-associatifs. Contrairement aux autres règles, qui
    %n'empêchent pas le parsing d'un fichier mais en changent simplement
    %l'AST correspondant, l'ajout d'une règle \texttt{non-assoc} peut
    %empêcher le parsing : par exemple \texttt{++x++} ne pourra pas être
    %parsé (relativement à nos règles).
    \item Les règles qui définissent la déclaration de variables globales et la déclaration de fonctions entraient en conflit.
          Les règles étaient définies comme suit :
          \begin{itemize}
              \item $<type> <var>+,\texttt{;}$ où \emph{var} était définie par $<ident> | \texttt{*} <var>$
              \item $<type> \texttt{*}* <ident>\texttt{(} <argument>*, \texttt{);}$
          \end{itemize}
          Le problème est que pour reconnaître une déclaration de variable, l'automate doit réduire \texttt{*}*\emph{<ident>} en un \emph{<var>}, car si ce n'est pas fait directement un autre caractère sera empilé, or l'automate ne peut réduire qu'un ensemble de termes en haut de la pile. Mais alors, un fois ces termes réduits, l'automate ne peut plus reconnaître une déclaration de fonction, puisque \emph{<var>} ne figure pas dedans.

          Nous avons résolu ce conflit en réécrivant les règles en question.

      \item Nous avons été confrontés au conflit classique dû à la structure \texttt{if (…) … else …}.
            Le comportement par défaut de \texttt{menhir} sur ce conflit (favoriser la lecture, le \emph{shift}) est celui qui est attendu. Nous avons vérifié que l'arbre obtenu est bien le bon (voir section~\ref{subsec:html}).
      \item PROBLÈME AVEC GETS

\end{itemize}

\subsection{Étiquetage}

Pour étiqueter les nœuds de l'arbre syntaxique, nous avons déclaré un lexème paramétrable :

\begin{verbatim}
labeled(X):
   | x = X { (make_label $startpos $endpos), x }
\end{verbatim}

C'est un moyen assez peu intrusif d'étiqueter les nœuds, à condition bien sûr de mettre à jour le type de l'arbre.

Nous n'avons pas choisi de décorer les arbres avec un type
 enregistrement mutable comme proposé dans le cours sur le typage.
Chaque fois que l'on a du étiqueter des arbres, nous avons réutilisé la
même stratégie de renvoyer des couples étiquette*arbre.

\subsection{Vérification de la correction}
\label{subsec:html}

Nous avons mis en place un système de tests avec un script shell, mais il vérifie uniquement si le code de retour est bien celui qu'on attendait.
Dans le cas où l'analyse a réussi, cette vérification est insuffisante puisqu'il faut s'assurer que l'arbre construit est bien celui qu'on attendait.
C'est particulièrement important quand il s'agit de vérifier la gestion des priorités opératoires, par exemple.

Pour vérifier si l'arbre généré est bien l'arbre attendu, nous avons écrit un module qui fait la traduction inverse :
à partir d'un arbre syntaxique il écrit un code source.
Le code source généré est écrit en HTML, ce qui permet d'ajouter de la coloration syntaxique au passage.
Les étiquettes qui indiquent la position dans le source initial ne sont pas perdues :
il suffit de survoler un lexème pour voir s'afficher sa position.

Enfin, pour vérifier et débuguer le typeur, nous avons adapté le module
présenté plus haut pour qu'il affiche non plus la position mais le type
de l'expression survolée.
Plus précisément, quand on survole un caractère, le type de la plus petite expression contenant ce caractère s'affiche.
Quand on survole un nom de fonction, son prototype est affiché.

Pour utiliser ces modules, l'utilisateur doit rajouter en argument
\texttt{-htmlt} ou \texttt{-htmlp}, en fonction de l'étiquetage désiré.

Par défaut, l'afficheur génère très peu de parenthèses.
Pour vérifier le parenthésage d'une expression, on peut passer en \texttt{-lisp-mode}, qui force l'écriture de beaucoup de parenthèses.

\section{Typage}

%TODO
Au moment du typage, nous avons effectué un certain nombre de traitements
dans le but de simplifier la production de codes future:
\begin{itemize}
\item On différencie une déclaration de variables locales ou globales d'une
déclaration dans une \texttt{Struct} ou \texttt{Union}, que l'on appelle
alors un champ.
%DELETE ce qui suit?
\item Pour les déclarations dans les \texttt{Struct} et \texttt{Union} on
découpe les déclarations de variables. Autrement dit, \texttt{int a,b;}
devient \texttt{int a; int b;}
\item Lors de la génération de l'arbre typé, 
on vérifie qu'il existe une fonction main.
\end{itemize}

\appendix
\section{Conventions de nommage}

Les types commencent par une lettre indiquant dans quel contexte on les utilise :
\begin{itemize}
    \item \texttt{a} comme dans \texttt{aexpr} : type utilisé dans l'arbre de syntaxe abstraite pour représenter un nœud.
    \item \texttt{l} comme dans \texttt{lident} : équivalent étiqueté (par une position dans le source) du type précédent.
    On définit généralement \texttt{l\emph{type} = label * a\emph{type}}.
\item \texttt{w} comme dans \texttt{wexpr} : arbre dont la descendance est étiquetée par des types C
\item \texttt{t} comme dans \texttt{texpr} : équivalent étiqueté par un type C du type précédent, généralement défini par \texttt{t\emph{type} = expr\_type * w\emph{type}}.
\end{itemize}

Les constructeurs de chaque type sont préfixés par des majuscules indiquant quel type ils construisent :
\begin{itemize}
    \item \texttt{AU} : opérateur unaire dans un arbre de syntaxe étiqueté par des positions
    \item \texttt{AB} : idem pour un opérateur binaire
    \item \texttt{AE} : une expression dans l'AST (équivalent du \emph{<expr>} du sujet)
    \item \texttt{AI} : une instruction dans l'AST
    \item \texttt{ET} : un type qui peut être affecté à une expression (Expresion Type)
    \item \texttt{TE} : une expression typée (Typed Expression), pour construire l'AST typé
\end{itemize}

\end{document}


